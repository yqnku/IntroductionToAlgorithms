%*************************************************************************
%    > File Name: notes.tex
%    > Author: yqnku
%    > Mail: xiqian013@live.com 
%    > Created Time: 2015-11-21 23:50:45
%************************************************************************
\documentclass[UTF8,a4paper,12pt]{ctexart}
\usepackage{geometry}
\geometry{left=2.5cm,right=2.5cm,top=2.5cm,bottom=2.5cm}
\usepackage{hyperref}
\usepackage{clrscode}
\begin{document}
    \pagestyle{plain}
    \section*{\huge{notes}}
    \section{Foundations}
        \subsection{The Role of Algorithms in Computing}
            \subsubsection{Algorithms}
            \subsubsection{Algorithms as a technology}
            \subsubsection{Problems}
        \subsection{Getting Started}
            \subsubsection{Insertion sort}
                \textbf{Input}:A sequence of $n$ numbers $(a_1,a_2,\cdots,a_n)$.

                \textbf{Output}:A permutation $(a'_1,a'_2,\cdots,a'_n)$ of the input sequence such that $a'_1 \leq a'_2 \leq \cdots \leq a'_n$.

                \textbf{loop invariant}:We use loop invariants to help us understand why an algorithm is correct.We must show three things about a loop invariant:

                \textbf{Initialization}:It is true prior to the first iteration of the loop.

                \textbf{Maintenance}:If it is true before an iteration of the loop,it remains true before the next iteration.

                \textbf{Termination}:When the loop terminates,the invariant gives a useful property that helps show the algorithm is correct.

                \textbf{Pseudocode conventions}
            \subsubsection{Analyzing algorithms}
            \subsubsection{Designing algorithms}
                \textbf{The divide-and-conquer approach}:
                
                \textbf{Divide}:the problem into a number of subproblems that are smaller in instances of the same problem.

                \textbf{Conquer}:the subproblems by solving them recursively.If the subproblem sizes are small enough,however,just solve the subproblems in a straightforward manner.

                \textbf{combine}:the solutions to the subproblems into the solution for the original problem.

                \textbf{merge sort}:

            \subsubsection{Problems}      

\end{document}
